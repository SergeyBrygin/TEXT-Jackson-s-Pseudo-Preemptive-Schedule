\documentclass[10pt]{article}
\usepackage[dvips]{graphics,epsfig}
\usepackage[russian]{babel}
\usepackage{amsfonts,amssymb,amsmath}
\usepackage{cite}
\usepackage[utf8x]{inputenc}

\usepackage[german]{babel}

\textwidth 14cm
\textheight 20cm
\begin{document}

\noindent УДК 517.987 \hfill Вестник СПбГУ. Сер. 1, Т. , 2019 , вып. 3

\noindent MSC 28C99

\medskip

\noindent {\bf Об оценках обобщенной размерности Хаусдорфа. }

\medskip

\noindent {\it Г.А. Леонов, А.А.  Флоринский }

{\small
\noindent Санкт-Петербургский государственный университет, \\
Российская Федерация, 199034, Санкт-Петербург, Университетская наб., 7/9
}

\section{Введение}

 Проблема
\section{Предварительные определения и утверждения}
 Г.А.Леонова \cite{leonov} .


\paragraph*{Теорема 1.} 
Пусть множество $X$ и семейство $\Omega$ таковы, как описано выше. 
Следующие условия эквивалентны.

    1. Для всякой функции $h: \Omega \to [0, \infty]$ со свойством 
    $h(\Lambda) = 0$ функционал $\mu_{LH}$ является внешней мерой. 
    
    2. Функционалы $\mu_H$ и $\mu_{LH}$ совпадают  при любой функции 
    $h: \Omega \to [0, \infty]$ со свойством $h(\Lambda) = 0$.
    
    3.Выполнены следующие два утверждения.
    
        а) Семейство $\Omega$ диадически упорядочено отношением включения, 
        то есть для любых $A, B \in \Omega$ , выполнено $A \subset B $
        или $ A \supset B$ или $A \bigcap B = \Lambda$.
        
        б) Семейство $\Omega$ удовлетворяет условию обрыва возрастающих цепей, 
        то есть каждая возрастающая по включению последовательность элементов
        семейства $\Omega$ конечна.
        
    4.  Семейство $\Omega$ удовлетворяет условию (*). 
    
\paragraph*{Доказательство.}
Ясно


\section{Основные определения}

 Перейдем 
 
\begin{enumerate}
    \item Для любых $\Delta_1 \subset \Delta_2$
    из $\Omega$ выполнено
    $\aleph (\Delta_1) \subset \aleph(\Delta_2) $
    
    \item Для любой убывающей последовательности
    $\{\Delta_i\}_{i=1}^\infty \subset \Omega$
    выполнено
    $\overset{\infty}{\underset{l = 0}{\bigcap}}
    \aleph (\Delta_i) \neq \Lambda$ 
    
    \item Для каждого 
    $\Delta \in \Omega$
    справедливо условие
    $
    \underset{l \to \infty}{\lim}
    \frac{ | \tilde{\Omega}_l(\Delta) | }
         { | \Omega_l(\Delta) | } =
    1
    $,
    где
    $
    \tilde{\Omega}_l(\Delta) =
    \{\Delta' \in  \Omega_l : 
    \aleph (\Delta') \subset \Delta \}
    $
\end{enumerate}



\paragraph*{Теорема 2}

Пусть
$\mathfrak{A} = (X, \Omega, h) - $ ОРП
с конечным индексом компактности и верхней размерностью
$d > 0$.
Пусть $J \subset [0, d] - $ компакт.
Тогда существуют 
$\sigma  \in \uparrow (\mathbb{N}) $
и $E \subset X$ такие,
что $\mbox{Dim}_H (\mathfrak{A_\sigma}, E) = J$.


Вначале мы установим ряд 
вспомогательных утверждений.

\section{Вспомогательные построения}



\paragraph*{Замечание.}
Заметим, что в случае, когда
$f - $отображение заданное на открытом множестве
евклидова пространства, класса $C^1$,
такое, что
при некотором $s > 0$
значение сингулярной функции 
$\omega_s (d_x f) $
меньше $1$ при каждом 
$x \in E$,
причем $E$ компактно и
$f(E) = E$,
то некоторая степень $f$
является
$(\alpha, s) - $сжимающим отображением на $E$.
Это сразу следует из классической оценки в доказательстве теоремы Дуади - Остерле (см. \cite{scan_boch} ).


\bibliographystyle{unsrt}
\bibliography{Leonov_Florinsky}



\end{document}

